\clearpage
\subsection{Création d'un personnage}

\subsubsection{Race}
Choisissez la race de votre personnage parmi celles proposées.

\subsubsection{Traits}
\begin{rebelist}
	\item Votre héros débute avec un d4 dans chaque Attribut et dispose de 5 points avec lesquels il peut les augmenter. Augmenter un Attribut d’un type de 	dé coûte 1 point.
	\item Vous avez 15 points pour vos Compétences.
	\item Chaque type de dé investi dans une Compétence coûte 1 point jusqu’à égaler le dé de l’Attribut dont dépend la Compétence. Au-delà, le coût est de 	2 points.
	\item Le Charisme est égal au total des bonus ou des malus donnés par les Atouts ou les Handicaps.
	\item L’Allure est de 6 cases.
	\item La Parade est égale à 2 plus la moitié de la Compétence Combat.
	\item La Résistance est égale à 2 plus la moitié de Vigueur. Vous pouvez y ajouter le bonus de l’armure portée sur le torse pour accélérer le calcul en 	cours de partie mais ce bonus ne compte pas pour les autres parties du corps.
\end{rebelist}

\subsubsection{Atouts \& Handicaps}
Vous gagnez des points supplémentaires en prenant un Handicap Majeur (2 points) et jusqu’à deux Handicaps Mineurs (1 point chacun).

Pour 2 points vous pouvez :
\begin{rebelist}
	\item Augmenter un Attribut d’un type de dé.
	\item Choisir un Atout.
\end{rebelist}

Pour 1 point vous pouvez :
\begin{rebelist}
	\item Augmenter une Compétence d’un type de dé.
	\item Doubler vos fonds initiaux.
\end{rebelist}

\subsubsection{Équipement}
Débutez avec 500\crg.

\subsubsection{Histoire personnelle}
Rajoutez les détails concernant le personnage qui vous semblent importants.

\subsection{Progression}
Le rang détermine la puissance approximative de votre héros. Le nombre de points d’XP que vous avez gagné détermine le rang selon le tableau suivant : 

\begin{itemtable}[ l l ]
	\textbf{XP}		& \textbf{Rang} \\
	00-19 			& Novice \\
   	20-39 			& Aguerri \\
   	40-59 			& Vétéran \\
   	60-79 			& Héroïque \\
   	80+				& Légendaire
\end{itemtable}

Tout les 5XP votre héros bénéficie d’une progression. Lors d’une progression il peut :
\begin{rebelist}
	\item Choisir un nouvel Atout.
	\item Augmenter une Compétence dont la valeur est égale ou supérieure au Trait associé.
	\item Augmenter deux Compétences dont les valeurs sont inférieures aux Traits associés.
	\item Prendre une nouvelle Compétence à d4.
	\item Augmenter un Attribut d’un type de dé. Vous ne pouvez choisir cette option qu’une fois par Rang ou lors d’une progression sur deux après avoir atteint le rang Légendaire. Il n’est pas possible de progresser au-delà de d12 dans un Trait par le biais d’une Progression, mais vous pouvez consulter les Atouts Professionnel et Expert (p.~\pageref{sec:atout-professionnel}).
\end{rebelist}

\subsection{Fiche de personnage}
Pour finir avec la création de personnage, je vous propose cette fiche de perso, initialement pour \citetitle{savage-worlds} mais qui est tout à fait utilisable puisqu’on a collé au plus à l’ouvrage d’origine.

\cite{torgan-savage-perso}